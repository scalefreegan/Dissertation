% ========== Appendix B
 
\chapter{iGBweb: an interactive genome browser for the web}

\label{appendix:b}

\noindent Appendix B has been submitted:\\

\noindent Salvanha, DM, Brooks AN, Reiss DJ, V$\hat{e}$ncio, RZN, Baliga NS. iGBweb: an interactive genome browser for the web. Submitted.\\

\section{Abstract}

\textbf{Summary/Motivation}: iGBweb is a web tool that enables interactive visualization of systems-scale data in the context of the genome. iGBweb implements a suite of features for large-scale data exploration within an easy-to-embed, browser-independent web application. The application is open-source, highly-customizable and easily-integrated into existing HTML5 web applications. Multiple data types can be imported, rendered, paired, and synchronously animated by iGBweb, including: heat maps, line charts and a novel positional quantitative-string track. iGBweb is the first genome browser to allow direct visualization and animation of dynamic biological processes.\\

\noindent \textbf{Availability and Implementation}: The iGBweb front end is implemented using HTML5 resources (e.g. AJAX, D3.js, SVG, CSS). A Java and MySQL back-end web-service is also available for data retrieval if necessary. Source code (LGPL license), documentation, and examples are available at: \href{http://igbweb.systemsbiology.net}{http://igbweb.systemsbiology.net}.

\section{Introduction}

Biological systems can now be studied across multiple scales by integrating and modeling diverse kinds of molecular measurements. The high-throughput technologies used in these studies generate very large amounts of complex data. Effective representation, integration, and interactive analyses are critical to extract biological insight from these data. We describe a tool that allows end-users to interact with integrated visualizations of dynamic biological processes in the context of the genome. 

iGBweb is an interactive web application module that integrates biological data using multiple genomic tracks. The user interface consists of a minimal Javascript library that provides a straightforward web environment for developers and end-users alike. Data can be uploaded to iGBweb using several strategies, enabling rapid development. Unlike other genome browsers,  iGBweb can animate paired data from experiments with a dynamic context (e.g., changes in transcript expression levels and transcription factor binding across a time-course). iGBweb is also readily embedded in pre-existing web resources, encouraging developers to extend their own web approaches using our tool.

\section{Implementation}

iGBweb is a client-side application with an optional server-side component that leverages modern web technologies. The iGBweb front-end is cross-platform, browser independent, and readily combined with other web technologies. The front-end user interface collects data from a back-end data repository, rendering the data as track(s) in the browser. The front-end module was implemented using the D3.js library (Data-Driven Documents; \cite{bostock_d3_2011}). The UI was implemented as independent modules to provide scalability and modularity for web integration (including genome, focus, and context viewers, as well as view controllers). Since the modules are semi-independent, they can be added or re-implemented as necessary. New track types can be developed easily as well. 

iGBweb requires a server-side implementation for data retrieval. Developers can use their own back-end (as simple as a structured JSON file) or our pre-defined back-end template. This optional server-side component uses Java (JAX-RS) to expose RESTFul retrieval services that consume a MySQL database. Alternatively, developers can connect to their own database using Ajax calls. 

\section{Available Features}

\subsection{Features}

iGBweb provides a suite of features to represent diverse biological data types along a genomic axis. The iGBweb software represents genome-anchored data as two-dimensional graphics, including heat maps, bar charts, data points and segments. In addition to encoding a suite of diverse graphical representations, iGBweb also includes an intuitive API that encourages development of additional features to enhance end-user productivity. All of the graphical properties (including track style attributes such as color, symbol, and scale) are customizable and can be programmatically changed, if necessary. Each of these design features was implemented in iGBweb to make it easy for developers to implement a user-friendly genome browser that can be customized to meet individual project requirements. 

Beyond bringing standard genome browser visualizations to the web, iGBweb includes two features that are not currently supported by other genome browsers. First, iGBweb can represent scaled, positional quantitative strings. It can map any ASCII character to genomic coordinates. This feature facilitates representation of data abstractions, such as position-specific scoring matrices that are often used to represent DNA or protein sequence motifs and dot-bracket notation to encode 2D RNA structure (Use case 1). Second, iGBweb can animate data transitions. This feature is particularly useful to facilitate analysis of disparate data sets (e.g. ChIP-chip, microarray or time-series), which can be combined, paired, and rendered synchronously, allowing the end-user to visualize connections and correlations in their data (Use case 2). The software includes data transition modes for four track-types, including: heat maps, quantitative positional, quantitative segment and quantitative strings. Both of these novel features aim to modernize the genome browser, making it a core tool for hypothesis generation in systems biology.  

\subsection{Use cases}
	 	 	
To illustrate the flexibility, simplicity, and utility of the iGBweb we describe three use cases. The examples are presented in order of increasing complexity, demonstrating the full range of capabilities offered by iGBweb. The first example integrates quantitative gene expression data with character representations of RNA secondary structure predictions; the second pairs ChIP-chip and gene expression data to explore the effects of a transcription factor knockout; finally, the third demonstrates how easy it is to extend pre-existing web applications using iGBweb. Each example reveals how iGBweb enables intuitive visualizations to be developed quickly and easily. Online tutorials are available to guide the developer through configuration of iGBweb for examples 1 and 2, including all steps required to implement iGBweb, from reading structured data into the genome browser to configuring tracks to display the data.

\subsubsection{Use case 1.} One of the novel features of iGBweb is its ability to render genomic data as scaled and/or dynamically modulated ASCII characters. For the first example, we visualized data from a study by Randau \cite{randau_rna_2012} that revealed an abundance of processed small RNAs (sRNAs) in the hyperthermophilic archaeon Nanoarchaeum equitans. RNA secondary structures were displayed by iGBweb in standard dot-bracket notation. We visualized the authors’ RNA-seq data together with RNA structure predictions from all 39 families in the Rfam database, and structures for tRNAMet and tRNAVal from the paper. In this case, simultaneous visualization of both data types not only revealed the abundance of sRNAs noted by Randau, but also highlighted potentially interesting correlations between RNA secondary structure and expression levels. This complete example can be fully developed in less than one hour. 

\subsubsection{Use case 2.} The second example illustrates how iGBweb enables intuitive, dynamic visualization for paired measurements. This example includes data integration (as in Use case 1.), as well as animation to visualize paired transitions across two kinds of data. We paired ChIP-chip binding data for the E. coli transcription factor PurR in two conditions (+/- supplemental adenine) with gene expression measurements for both wild type and ΔPurR mutants in these conditions (GEO GSE26591; \cite{cho_purr_2011}). The data are visualized in iGBweb as two genome tracks: a heatmap for gene expression; and quantitative segments for ChIP-chip data, including vertical bars to indicate TF binding sites. Integrated, visual representation allows users to detect important patterns in their data; for example, at the \textit{carA} promoter (a gene involved in pyrimidine biosynthesis) the visualization clearly reveals that PurR binding in the presence of adenine results in repression of \textit{carA}. This type of visualization can easily be extended to other types of experiments with multiple data types and/or dynamics, including time course studies. 

\subsubsection{Use case 3.} As a last example, we demonstrate how easy it is to plug the iGBweb into preexisting web apps. We implemented iGBweb to facilitate exploration of a computational model that predicted the condition-dependent magnitude of influence of gene regulatory elements (GREs) at each nucleotide residue across the genome (position-based numeric values; EGRIN 2.0 [Brooks AN, DJ Reiss, et al. Accepted MSB]). These predictions were contained in a PostgreSQL database with an associated Django web app. iGBweb was embedded into this preexisting web framework using Javascript to replace static representations of the model predictions. Given a gene name, iGBweb queries the EGRIN 2.0 database to retrieve information about the gene, including model predictions about the gene promoter region. Using this information, iGBweb produces a line plot with a dynamic representation of changes in predicted GRE-associated influences across that region. The implementation allows the user to cycle through condition-dependent predictions and select which GREs are plotted. In addition, the user can activate the novel iGBweb positional quantitative-string feature to represent the genomic sequence within the line plot. All of these features were easily customizable on the front-end. Implementation of iGBweb in this large web resource took less than one week (including extensions to the default iGBweb features). 

\subsection{Documentation}

The  iGBweb  source code is freely available at  \href{http://igbweb.systemsbiology.net}{http://igbweb.systemsbiology.net}. The site includes tutorials for the Use cases, API documentations and examples source codes that can be used as a template.

\subsection{Future Directions}

As genomic technologies evolve, the ability to adapt data visualization platforms for new data types will become increasingly important. We propose a public track repository that will allow iGBweb developers to keep up with the most recent trends in visualization technologies using iGBweb. The resource will enable developers to contribute track implementations, tutorials, as well as data. This, in turn, will enable a community of developers to create user-friendly, flexible, ‘plug-and-play’ genome browsers that will advance biological understanding. 