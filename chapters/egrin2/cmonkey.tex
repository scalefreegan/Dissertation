\subsection{\cm: integrated biclustering algorithm, updated for ensemble analysis}

\subsubsection{Introduction and summary}

The \cm\ integrated biclustering algorithm was described and fully
benchmarked in \cite{Reiss2006n}. In short, the algorithm computes
putatively co-regulated modules of genes over subsets of experimental
conditions from gene expression data, constrained by information
provided by genome sequence (\textit{de novo} identification of
conserved \textit{cis}-regulatory motifs in gene promoters), and functional
association networks. Its defining characteristic is that it
combines all three types of data (expression, sequence and networks)
together into an integrated model that uses a stochastic
optimization procedure to identify modules that best satisfy all three
constraints, simultaneously.

We refer the reader interested in details of the \cm\ data integration
model and optimization procedure to \cite{Reiss2006n}. Here, we
briefly summarize the procedure as it was utilized in the \egrine~
model construction. The \cm\ integrated biclustering algorithm
identifies groups of genes co-regulated under subsets of experimental
conditions, by integrating various orthogonal pieces of information
that support evidence for their co-regulation, and optimizing
biclusters such that they are supported by one or more of those
additional constraints. The three sources of evidence for
co-regulation leveraged by \cm\ to score gene clusters are (1)
tight co-expression in subsets of available gene expression
measurements (similarity of expression profiles); (2) quality of
\textit{de novo} detected \textit{cis}-regulatory motifs in gene
promoters (putative co-binding of common regulators); and (3)
significant connectivity in functional association or physical
interaction networks (co-functionality). The algorithm served as the
cornerstone for the construction of the first global, predictive
Environmental Gene Regulatory Influence Network (EGRIN) model for
\halo \cite{Bonneau2007}, and has now been applied to many additional
organisms (\eg, \cite{Yoon2013} and unpublished).

To run \cm\ as part of an ensemble-based inference approach required
significant updates to the \cm\ algorithm. These updates primarily
addressed computational inefficiencies that led to long runtime. The
primary algorithm modification in the new implementation is global
optimization (rather than the local, individual cluster optimization
utilized by the original procedure). Additional algorithm updates
include changes to the individual scoring scheme for subnetwork
clustering, as well as integration of the scores. All of these changes
improved the procedure's runtime without significantly affecting the
algorithm's performance.

\subsubsection{Updates since original publication}

For incorporation into the \egrine~ensemble analysis, the \cm\
procedure and software was overhauled to improve runtime performance
and decrease memory usage. These modifications did not quantifiably
affect overall bicluster quality. Changes to the algorithm (and
parameters used for \egrine~construction) relative to the earlier
version described in \cite{Reiss2006n} are as follows:

1. Iteratively re-weighted constrained logistic regression
to determine gene/condition probabilities for bicluster membership was
replaced with a non-parametric cumulative distribution function on
gene/condition scores. Since the non-parametric function does not need
to be re-weighted, it is significantly faster to compute.

2. Rather than constraining the number of bicluster assignments per
gene/condition using a probability distribution, \cm\ now assigns a
fixed number of biclusters to each gene/condition, per run (a
user-defined parameter). For this study the parameter was set to 2 for
genes, and to $k/2$ for conditions, where $k$ is the total number of
biclusters in the run, also a user-defined parameter. This
modification effectively altered the bicluster optimization from a
local problem (single bicluster) with limited cross-bicluster
constraints to a global problem, similar in principle to the widely
used $k$-means clustering algorithm.

3. Since \cm\ uses the updated constraint (of 2; see above) to choose
the two ``best'' biclusters for each gene (and likewise the best $k/2$
biclusters for each condition), there is no sampling as in the prior
version. Instead, stochasticity is added to prevent the optimization
from falling into local minima. The algorithm allows at most one
change in bicluster assignment per gene/condition, per iteration. This
is accomplished by adding a small amount of noise to each
gene/condition's bicluster membership weight. The noise occasionally
allows moves that decrease a bicluster's total score. This noise
decreases towards zero as the number of iterations increases.

4. Motif detection is the most computationally expensive part of the
procedure. To circumvent significant computation time, we limit motif 
detection to every 100 iterations (for a typical,
2,000 iteration \cm\ run). During the 99 iterations between motif
searches, the biclusters are optimized to contain instances of those
detected motif(s). We found that this does not impair the ability of
\cm\ to detect motifs.

The overall effect of these changes (in addition to other minor
modifications and improvements) resulted in an algorithm run-time
reduction of about 4-fold. This, in turn, enabled \cm\ to be run
numerous times on a modest 8-core compute node (\eg, a
\tmsamp{c1.xlarge} Amazon EC2 node) in under six hours per run
(compared to several days to a week for the original
\cm). Practically, the effect of these modifications to the
algorithm resulted in a single \cm\ run generating fewer duplicate
biclusters, primarily because each gene is allowed to be a member of
only two biclusters. We found that this increased the overall
diversity of regulation discovered by each \cm\ run
(condition-specific gene clusters and corresponding
\textit{cis}-GREs).

\subsubsection{Detailed \cm\ algorithm description}

The \cm\ algorithm initiates by seeding $k$ biclusters, typically using
the simple, widely-used and effective $k$-means clustering on the
input expression data set. \cm\ itself performs a global optimization,
in many ways similar to the $k$-means clustering algorithm, which we
used as a model. After beginning with an initial assignment of each
gene into $k$ clusters and a chosen distance metric, the basic
$k$-means algorithm iterates between two steps until convergence: (1)
(re-)assign each gene to the cluster with the closest centroid and (2)
update the centroids of each modified cluster. The updated \cm\ 
algorithm performs an analogous set of moves with four primary
distinctions: (1) the distance of each gene to the ``centroid'' of
each cluster is computed using a measure that combines
condition-specific expression profile similarity, $cis$-regulatory
motif similarity, and connectedness in one or more gene association
networks; (2) each gene can be (re-)assigned to more than one cluster
(default 2); (3) at each step, conditions (in addition to genes) are
moved among biclusters to improve their cohesiveness; and (4) at each
step, genes and conditions are not always assigned to the most
appropriate clusters. We now elaborate upon these four details.

\cm\ begins each iteration with a set of bicluster memberships
$\{m_i\}$ for each element (gene or condition) $i$, where by default
$\|m_i\| = 2$ for genes and $\|m_i\| = N_c/2$ for conditions ($N_c$ is
the number of conditions, or measurements, in the expression data set;
note that for standard $k$-means clustering, $\|m_i\| = 1$ for genes
and $\|m_i\| = N_c$ for conditions). \cm\ then computes score matrices
(log-likelihoods, in practice) $\mathbf{R}_{ij}$, $\mathbf{S}_{ij}$,
and $\mathbf{T}_{ij}$, for membership of each element $i$ in each
bicluster $j$, based upon, respectively, co-expression with the
current gene members ($\mathbf{R}$), similarity of motifs in gene
promoters ($\mathbf{S}$), and connectivity of genes in networks
($\mathbf{T}$). For the network scores ($\mathbf{T}$), the originally
published procedure \cite{Reiss2006n} computed a $p$-value for
enrichment of network edges among genes in each bicluster using the
cumulative hypergeometric distribution. This computation was
inefficient, and moreover could not account for weighted edges in the
input networks, so we replaced it with a more standard weighted
network clustering coefficient \cite{Watts1998}, evaluated only over
the genes within each bicluster.

Following computation of the individual component scores,
\cm\ computes a score matrix $\mathbf{M}_{ij}$ containing the
integrated score (a weighted sum of log-likelihoods) supporting the
inclusion of gene $i$ in bicluster $j$. At this stage the updated
version of \cm\ then computes a cumulative density distribution from
each bicluster's $\mathbf{M}_{\cdot j}$ to obtain a posterior
probability distribution $p_{ij}$, that each element $i$ should be in
each cluster $j$, which is used to classify cluster members based upon
these scores. The width of the density distribution kernel is set
dynamically to be larger for smaller (fewer gene) biclusters, so as to
increase the tendency to add genes to small biclusters, rather than to
remove them. In the updated procedure, we then add a small amount of
normally-distributed random ``noise:'' to the scores
$\mathbf{M}_{ij}$, in order to achieve a similar type of stochasticity
to the original version of the algorithm (which was originally
obtained using sampling, and which helps prevent the algorithm from
falling into local minima; this noise decreases during the run to zero
at the final iteration). The result of this noise is that at the
beginning of a \cm\ run, biclusters are rather poorly defined
(co-expression, for example, is weak), but during the course of a full
set of 2,000 iterations, as this noise is decreased, the biclusters
settle into minima.

Finally, at the end of each iteration, \cm\ chooses a random subset of
elements (genes or conditions) $i$, and moves $i$ into bicluster $j$
if, for any biclusters $j'$ which it is already a member, $p_{ij} >
p_{ij'}, \forall j'$, and out of the corresponding worse bicluster
$j'$ for which $p_{ij} > p_{ij'}$. Thus, as with the $k$-means
clustering algorithm, the updated \cm\ performs a global optimization
of all biclusters by moving elements among biclusters to improve each
element's membership scores.

\subsubsection{Parameter ranges used for \egrine}

The default values for all additional parameters used for \cm, and for
\MEME\ (which is used by \cm\ for motif detection; \cite{Bailey1998}),
are the same as those itemized in \cite{Reiss2006n}. These defaults
were used exclusively for all \halo~ \cm~ runs. These default
parameters are itemized in Table~\ref{tab:cmparams:halo}. The only
parameter that varied from run-to-run for the \halo~ \cm~ runs was the
number of conditions (columns in the expression matrix) included. As
\cm~development was occurring in parallel to development of the
\egrine~modeling approach (primarily involving bug fixes), we also
list the official \cm~version number(s) used.

\vspace{.2cm}
\begin{table}[h!]
\begin{tabular}{|l|l|l|r|} 
\hline
Parameter              & Value              & Note \\ \hline
Version               & 4.5.4(174), 4.6(191), 4.6.2(109)        & \cm~package versions (and number of runs) used \\
$N_{\mathrm{conds}}$    & 242:300          & Number of conditions included (range) \\
$k$                   & 250              & Number of biclusters \\
$N_{\mathrm{gene}}$    & 2          & Number of biclusters per gene \\
$N_{\mathrm{iter}}$     & 2000               & Number of iterations \\
$w_{\mathrm{max}}$     & 24               & Maximum \MEME~motif width \\
$w_{\mathrm{min}}$     & 6               & Minimum \MEME~motif width \\
$l_{\mathrm{search}}$     & -150 -- +20       & \MEME~search distance from gene start \\
$l_{\mathrm{scan}}$     & -250 -- +30       & \MEME~scan distance from gene start \\
$n_{\mathrm{motif}}$     & 2       & Number of \MEME~motifs searched per bicluster \\
$s_{\mathrm{r}}$     & 1       & Scaling factor for expression scores \\
$s_{\mathrm{m}}$     & 1       & Scaling factor for motif scores \\
$s_{\mathrm{n}}$     & 0.5       & Scaling factor for network scores \\
$w_{\mathrm{op}}$     & 0.5       & Relative weight for operon network \\
$w_{\mathrm{string}}$     & 0.5       & Relative weight for STRING network \\
\hline
\end{tabular}
\caption{\cm~parameters used for the \halo~ensemble.}
\label{tab:cmparams:halo}
\end{table}
\vspace{.2cm}

\noindent 
In contrast to the \halo~ runs, for the \eco~ runs, we varied 
several parameters randomly between the ranges itemized in Table~\ref{tab:cmparams:eco}.

\vspace{.2cm}
\begin{table}[h!]
\begin{tabular}{|l|l|l|r|} 
\hline
Parameter              & Value              & Note \\ \hline
Version               & 4.9.0(106)        & \cm~package versions (and number of runs) used \\
$N_{\mathrm{conds}}$    & 181:279          & Number of conditions included (range) \\
$k$                   & 352:547              & Number of biclusters \\
$N_{\mathrm{gene}}$    & 2          & Number of biclusters per gene \\
$N_{\mathrm{iter}}$     & 2000               & Number of iterations \\
$w_{\mathrm{max}}$     & 12:30               & Maximum \MEME~motif width \\
$w_{\mathrm{min}}$     & 6               & Minimum \MEME~motif width \\
$l_{\mathrm{search}}$     & -(100:200) -- +(0:20)       & \MEME~search distance from gene start \\
$l_{\mathrm{scan}}$     & -(-150:250) -- +(0:50)       & \MEME~scan distance from gene start \\
$n_{\mathrm{motif}}$     & 1:3       & Number of \MEME~motifs searched per bicluster \\
$s_{\mathrm{r}}$     & 2:4       & Scaling factor for expression scores \\
$s_{\mathrm{m}}$     & 0.5       & Scaling factor for motif scores \\
$s_{\mathrm{n}}$     & 0.5       & Scaling factor for network scores \\
$w_{\mathrm{op}}$     & 0:1       & Relative weight for operon network \\
$w_{\mathrm{string}}$     & 0:1       & Relative weight for STRING network \\
\hline
\end{tabular}
\caption{{\cm}~parameters used for the \eco~ensemble.}
\label{tab:cmparams:eco}
\end{table}
\vspace{.2cm}

\subsubsection{\cm\ software availability}

The \cm\ software is available as an open-source \tmsamp{R} package
\cite{Ihaka1996}.  With this package the algorithm can be easily
applied to nearly any sequenced microbial species (given user-supplied
expression data). The package automatically downloads and integrates
genome and annotation data from various external sources, including
\tmsamp{RSA-tools} \cite{vanHelden2000}; \tmsamp{Microbes Online}
\cite{Alm2005}; and \tmsamp{EMBL STRING}
\cite{Szklarczyk2011}. Additionally, the package can generate
interactive web-based and \tmsamp{Cytoscape} output
\cite{Shannon2003}, allowing users to explore the resulting modules
and motifs in the context of external data, software, and databases
via the \tmsamp{Gaggle} \cite{Shannon2006}. Examples of automatically
generated output are available at the \cm\ web site. Supplementary $R$
packages with example expression data for organisms including
\halo\ and \eco\ are also available from the \cm\ website.

