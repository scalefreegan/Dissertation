\subsubsection{Genome sequence data and annotatations for \cm~analysis}

We used genome sequences and gene annotations (coding regions)
collated in \tmsamp{RSA-tools} \cite{vanHelden2000} for both
organisms in this study (\halo~and \eco). These data were themselves
collated to annotate regulatory sequences of all sequenced genomes in
\tmsamp{RefSeq}. Rather than using the \tmsamp{RSA-tools}-annotated
promoter regions, we computed them ourselves as regions (-250 nt to
+50 nt) surrounding the annotated translation start site of each
gene/operon (see below for operon annotations). 

In all cases where probe identifiers in the mRNA expression compendia
used for this analysis could not be directly matched to gene
annotations (or operon predictions or functional associations; see
below), we used the \tmsamp{RSA-tools} ``\tmsamp{feature\_names.tab}''
table of identifier synonyms to perform the match. In cases where the
match was still not possible, we excluded the probe/ annotation/
association from analysis.

\subsubsection{Operon membership predictions used for \cm~analysis}

We used operon predictions for both \halo~and \eco~predicted by
\cite{Price2005a} from the \tmsamp{Microbes Online}
database \cite{Alm2005}. These predictions are updated regularly. The predictions are based upon
genomic proximity and co-expression in publicly-available microarray
data compendia. We used the versions downloaded
from the website as of March, 2009. These included
predicted operon memberships for 826 genes in \halo~ and for 2,639
genes in \eco.

\subsubsection{Predicted transcriptional regulators used for \nwinf~analysis}

\paragraph{\halo}

For \halo, we used the same set of putative transcription factors
(TFs) as \cite{Bonneau2006,Bonneau2007}. This list of 124
regulators was selected from among the 2,400 \halo\ genes which are
annotated as known or putative TFs based upon sequence or predicted
structural homology \cite{Bonneau2004}.

\paragraph{\eco}
\label{section:eco_tfs}

To enable direct comparison of our results to DREAM5, we used the list
of 296 putative \eco~ transcriptional regulators collated
by \cite{Marbach2012}. Their list was obtained by combining the list
of TFs defined by RegulonDB \cite{Gama-Castro2011} with TFs identified
using Gene Ontology (GO) terms: \textit{biological process} terms
related to transcription (\texttt{GO:0009299;mRNA transcription}
or \texttt{GO:0006351;transcription, DNA dependent}) and
GO \textit{molecular function}
\texttt{GO:0003677;DNA binding} or any child terms.

\subsubsection{Functional association networks integrated into \cm~analysis}

We used EMBL STRING \cite{Szklarczyk2011} v9.0 database
of predicted functional associations between genes for both organisms
(\halo~and \eco) to constrain module construction in \cm, as described
below. The confidence scores estimated by \cite{Szklarczyk2011} were
incorporated into the \cm~constraints. These
networks included 151,826 associations among 2,559 genes in \halo, and
878,972 associations among 4,136 genes in \eco.

