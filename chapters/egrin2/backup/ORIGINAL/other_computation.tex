\subsection{Functional enrichment estimates for genes in corems}

We computed functional enrichment for genes organized into corems
using DAVID (Dennis et al., 2003) and the DAVIDQuery R-package (Day
and Lisovich, 2010). Enrichments for each corem are available on the
web site.

\subsection{Conditional co-regulation of genes organized in corems}

We defined the conditions in which genes in a corem were co-regulated
as the set of experiments in which the genes of a corem are more
tightly co-expressed than one would expect at chance. We statistically
evaluated tight co-expression using relative standard deviation (RSD 
$=|\sigma/\mu|$) and resampling. We chose RSD (rather than, for example,
standard deviation, $\sigma$) to avoid over-weighting conditions in which the
mean relative expression is close to zero. The significance of an RSD
value for a given condition relative to each corem was estimated
by resampling: for a corem with $k$ gene members, and for each
condition, $c$, we computed at least 20,000 RSD values for $k$ randomly
sampled expression measurements in $c$, to determine the likelihood that the
observed co-expression has lower RSD than expected by chance ($p$-value
$< 0.01$). The resampling procedure resulted in condition sets for
corems that contained from 1.4\% to 85.5\% of the conditions in
\halo\ and 7.9\% to 66.6\% conditions in \eco.

\subsection{Conditionality of GRE influence}

The upstream promoter regions of most genes contain multiple EGRIN
2.0-predicted GREs (\eg, carA in Figure 2). A key insight of our
model is that not all of these sites are equally important for
controlling gene expression in all experimental conditions. We refer
to changes in the relative influence of GREs across conditions as
“conditional activity” of GRE elements. Although, to be clear, we do
not imply that the transcriptional activity at a GRE is attributable
to the DNA sequence itself, but rather the TF that binds to that
sequence in particular environments. We leveraged the GREs discovered
in genes grouped into corems and the conditional co-expression of
those groups of genes to predict conditionally active GREs in EGRIN
2.0.

Specifically, to discover active GREs for each corem we combined
predictions from (1) genome-wide motif scans (Section 5 above) that
predict the GRE locations in an expanded region around each gene’s
promoter in the corem using all of the ensemble predictions (1,000 nt
window: -875 nt upstream to 125 nt downstream), and (2) the conditions
discovered in biclusters that are most representative of the corem
(\ie, containing the largest fraction of genes from the corem, top
decile). GREs that occurred frequently in these biclusters were
considered putatively responsible for co-regulating the set of genes
in the condition-specific context of the corem (q-value ≤
0.05). Finally, we computed the average distances of all GREs to the
start codons of each gene in the list (collapsing sites if they
occurred within 25 nt of one another). The precise locations of all
GREs for the {\it H. salinarum} dpp operon-related corems (Figure 3) are
listed in Table E8, while the locations for GREs involved in
conditional modulation of the PurR regulon (Figure 4) are provided in
Table E9.

We represented the active GREs upstream of a gene or within a corem as
a pie chart, showing the normalized frequency with which the GREs
computed above occurred in biclusters containing that gene. For
example, if GREs 1, 2, and 3 occurred in 25, 50, and 200 biclusters
containing gene A, the pie chart for gene A would have sectors of area
0.09, 0.18, and 0.73 respectively. For corems, we computed the
normalized frequency of GREs for all genes of the corem. For example,
if GREs 1, 2, and 3 occurred in promoters of 10, 10, and 20 of the
genes of the corem, their areas would be 0.25, 0.25, and 0.5
respectively.

\subsection{Detection of conditional operons}

Conditional-specific transcriptional isoforms of operons were
predicted through corem membership. Specifically, if any of the genes
in an operon were found in a corem that did not contain all the other
genes of the operon, we predicted that the operon had conditional
isoforms. Operon annotations for both {\it H. salinarum} and {\it E. coli} were
derived from MicrobesOnline. All predicted conditional operons,
including the specific break sites and transcriptional isoforms is
available on the website. The full list of validated predictions is
provided in Table E7.

\subsection{Environmental ontology construction and usage}

We recorded a rich meta-data set for all 1495 experiments conducted
for {\it H. salinarum}. The meta-data includes a detailed description of
each experiment, including, for example: media composition, genetic
background, concentration of perturbant, internal reference batch id,
person who conducted the experiment, etc. We used this
meta-information to classify experiments in an ontological framework,
where two experiments can share specific meta-descriptions (\eg, 1e-3
mol/L EDTA), or inherit more general relationships from the
ontological structure (\eg, chemical perturbation). We used OBO-edit
to construct the ontology. The ontology contained 198 terms organized
across three primary branches (environmental state, experimental
state, and genetic state). The ontology flat file is available for
download and meta-data annotations for every array in the dataset are
available online.

We used the ontology to classify enriched environmental features for
GREs and corems (Figures 3-4). For corems, we used the set of
conditions in which genes in the corem are significantly co-expressed
(see 9 above) to compute term enrichment using the ontoCAT
R-package. Term enrichment was assessed statistically and reported as
q-values using the hypergeometric test with Benjamini-Hochberg
correction for multiple hypothesis testing.
