\subsubsection{Introduction and summary}

Procedures to infer a single Environment and Gene Regulatory Influence
Network (EGRIN) model from global data using cMonkey and Inferelator
were described previously (Bonneau et al., 2007; Bonneau et al., 2006;
Reiss et al., 2006). The EGRIN 2.0 modeling procedure updates this
process by applying modified cMonkey and Inferelator algorithms
(described above) repeatedly to subsets of the available expression
data. The subsets were selected semi-randomly, with available
biological information constraining the selection procedure by
including whole groups of related experiments when one was randomly
selected. For {\it H. salinarum}, we used manually curated metadata about
each experiment to group related experiments. For {\it E. coli} data set, we
did not have readily available metadata, so we instead grouped the
conditions based upon individual experiments (\eg, time series).
EGRIN 2.0 inference methodology is an ensemble learning approach, more
specifically a form of bootstrap aggregation (Breiman, 1996), or
sub-bagging.

Advantages of sub-bagging include its simplicity (basic model
averaging), its capability to reduce variance of the overall model
compared to individual runs (Bühlmann and Yu, 2002), and to avoid
overfitting (Krogh and Sollich, 1997). The power of an ensemble
learning approach results from its ability to “average out” errors in
the individual models. In the case of GRN inference, this feature
helps to overcome artifacts due to both experimental and algorithmic
noise. An incorrect classification from a single model instance is
identifiable because it re-occurs infrequently in subsequent runs
(assuming it is not the result of a systematic error in the algorithm
itself).

Sub-bagging of experimental conditions allows the model to up-weight a
restricted set of conditions for each individual EGRIN model in the
ensemble. This forces each EGRIN to model regulatory behaviors present
within a more narrow range of conditions. As a result, individual
EGRIN models have the opportunity to discover features (\eg,
conditions, genes, GREs) that may distinguish highly related responses
or occur in a very limited number of conditions in the data set. To
test this assumption, we performed a separate ensemble analysis of 30
EGRIN models that were generated using the entire {\it H. salinarum} data
set of 1,495 conditions. Across these 30 models, we expected to detect
$~20$ instances of the Bat GRE based on its occurrence in EGRIN 2.0
(\ie, motifs similar to GRE \#22; Figure~E\ref{fig:motclust}; (Baliga et al.,
2001)). Surprisingly, we did not detect a single GRE that matched the
Bat GRE (data not shown) when all conditions were included in each run
of the model. This is likely because the anoxic conditions under which
the TF that binds this GRE (Bat) is active represent only a small
portion of the entire data set.

Ensemble-based approaches are being used more frequently in biological
data analyses, including random forests (\ie, bags of decision trees)
for classifying genetic precursors to cancer (Breiman, 2001) and the
recently-published DREAM5 community ensemble of regulatory network
predictions (Marbach et al., 2012), which we used as a benchmark in
this manuscript to evaluate EGRIN 2.0 predictions. In principle, our
approach is similar to the stochastic LeMoNe algorithm (Joshi et al.,
2009), which uses Gibbs sampling to learn ensembles of regulatory
modules from gene expression data. EGRIN 2.0 is distinguished from
LeMoNe and similar algorithms by its ability to predict
transcriptional control mechanisms (\ie, GREs) and the conditions in
which they operate, both globally and within individual gene
promoters.

To construct and mine the EGRIN 2.0 ensemble we utilized additional
model aggregation and compilation procedures, including (1) motif
clustering (van Dongen and Abreu-Goodger, 2012) and scanning (Bailey
and Gribskov, 1998); (2) association network construction and backbone
extraction (Serrano et al., 2009); and (3) network community detection
(Ahn et al., 2010). These methods were used to identify GREs and their
genome-wide locations, gene-gene co-regulatory associations, and
corems, respectively. Each of these procedures is described in detail
below.

\subsubsection{``Ensemble of EGRINs'': generation and storage}

\subsubsection{Clustering of cis-regulatory motifs to identify GREs}

Each cMonkey bicluster contains at least one MEME-detected (Bailey
and Gribskov, 1998) de novo cis-regulatory motif. These motifs are
used by cMonkey to guide bicluster optimization (in addition to other
scoring metrics). There were 86,167 and 269,770 motifs detected across
the entire ensemble for {\it E. coli} and {\it H. salinarum}, respectively. Each
motif was represented in the model as a position-specific scoring
matrix (PSSM). To determine which of these motifs represented bona
fide GREs (as opposed to false positives), we computed pairwise
similarities between all motifs using Tomtom (Gupta et al., 2007)
(Euclidean distance metric; $q$-value threshold of 0.01 and overlap of 6
nt) and clustered the most highly similar PSSM pairs using mcl (Van
Dongen, 2008). The Tomtom motif similarity $p$-value threshold and the
mcl inflation parameter ($I$) were selected to (1) maximize the density
of edges between PSSMs inside clusters relative to the edges between
clusters, and (2) ensure that the mcl “jury pruning synopsis” was at
least 80 (out of 100). Criterion (1) aims to find a clustering that is
as inclusive as possible, while minimizing over-clustering, while (2)
is a built-in mcl metric that evaluates the quality of the clusters
resulting from the user-selected pruning strategy (I). The final
parameters were $p$-value cutoff = $10^{-6}$ and mcl $I = 4.5$ for {\it H. salinarum}
ensemble and $p$-value cutoff = $10^{-5}$ and mcl $I = 1.5$ for the {\it E. coli}
ensemble. We did not filter the motifs by $E$-value or other intrinsic
motif quality metrics; rather we enforced a cluster size threshold to
ensure that GREs were detected consistently. Clusters containing at
least 10 PSSMs were considered GREs (representing individual bicluster
detection instances). This resulted in 135 GREs for {\it H. salinarum}
(representing 27,991 motif instances, Table E2) and 337 for {\it E. coli}
(representing 12,773 motif instances, Table E3). Finally, we computed
a ``combined PSSM'' for each cluster as the unweighted mean of aligned
PSSMs within each cluster (Figure~2E; Figure~E\ref{fig:motclust}).

\subsubsection{Genome-wide scanning of motifs to obtain GRE locations}

We used scanning to discover GRE locations that were missed by the
rigid definition of a promoter in cMonkey (-250 to +50 nucleotides
surrounding the translation start site). This procedure was critical
for discovering GREs in non-canonical locations, such as internal to
operons.. To discover likely GRE locations throughout the genome, we
computed how well each GRE matched every position in the genome using
MAST (Bailey and Gribskov, 1998). Specifically, we recorded
significant matches for every PSSM in each GRE at each genomic
location subject to a position $p$-value threshold of $10^{-5}$. This
$p$-value cutoff corresponds to an expectation of discovering $~20$ sites
at random across the genome. For each GRE, we summed the number of
significant matches to each of the GRE’s PSSMs at each genomic
position. These counts were used to represent GRE composition in
promoters (Figures~2-3, Figures~S3-S5). In addition, we used these
scanned locations to identify GREs predominately located inside coding
regions. Since these GREs may be spurious (\eg, protein sequence
motifs) they were flagged, although they were not removed from our
global analysis.

\subsubsection{Statistical mining of the relationships in the ensemble}

Statistical associations between any entity in the EGRIN 2.0 ensemble
(\ie, genes, GREs, conditions, TFs; see Figure~1) can be evaluated
using the hypergeometric test for statistical enrichment. We use this
basic procedure to identify conditions associated with GRE influence,
and GREs associated with gene co-regulation, as we describe below.
