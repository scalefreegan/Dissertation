\subsection{cMonkey: integrated biclustering algorithm, updated for ensemble analysis}

\subsubsection{Introduction and summary}

The \cm\ integrated biclustering algorithm was described and
benchmarked in (Reiss et al., 2006). In short, the algorithm computes
putatively co-regulated modules of genes over subsets of experimental
conditions from gene expression data, constrained by information
provided by genome sequence (in the form of de novo identification of
conserved cis-regulatory motifs in the gene promoters), and functional
association networks. Its defining characteristic is that it
integrates all three types of data (expression, sequence and networks)
together into an integrated model that is optimized via a stochastic
optimization procedure to identify modules that best satisfy all three
constraints, simultaneously.

\subsubsection{Updates since original publication}

For incorporation into the EGRIN 2.0 ensemble analysis, the \cm\
procedure and software was overhauled to improve run-time performance
and decrease memory usage. These modifications did not quantifiably
affect overall bicluster quality. Changes to the algorithm (and
parameters used for EGRIN 2.0 construction) relative to the earlier
version described in (Reiss et al., 2006) are as follows:

1. The use of iteratively re-weighted constrained logistic regression
to determine gene/condition probabilities for bicluster membership was
replaced with a non-parametric cumulative distribution function on
gene/condition scores. Since the non-parametric function does not need
to be re-weighted, it is significantly faster to compute.

2. Rather than constraining the number of bicluster assignments per
gene/condition using a probability distribution, \cm now assigns a
fixed number of biclusters to each gene/condition, per run (this is a
user-defined parameter, and for this study was set to 2 for genes, and
to $k/2$ for conditions, where $k$ is the total number of biclusters in
the run, also a user-defined parameter). This modification effectively
alters the bicluster optimization from a local (single bicluster)
problem with limited cross-bicluster constraints, to a global problem,
similar in principle to the widely used k-means clustering algorithm.

3. Since \cm\ uses the updated constraint of (see 2, above) to
choose the two ``best'' biclusters for each gene (and the best $k/2$
biclusters for each condition), there is no sampling as in the prior
version. Instead, stochasticity is added, to prevent the optimization
from falling into a local minimum, by allowing at most one change in
bicluster assignment per gene/condition, per iteration, and by adding
a small amount of artificial noise to each gene/condition's
bicluster membership weight. This noise occasionally allows moves that
decrease a bicluster's total score, and the noise decreases to near
zero toward the end of the optimization.

4. The motif search, the most computationally expensive part of the
procedure, is limited to run every 100 iterations (for a typical,
2,000 iteration \cm\ run). During the 99 iterations between motif
searches, the biclusters are optimized to contain instances of those
detected motif(s). We found that this does not impair the ability of
\cm\ to detect significant motifs.

5. Finally, as part of the EGRIN 2.0 model construction, only the EMBL
STRING (v9) (Szklarczyk et al., 2011) set of predicted gene functional
associations, and predicted operons (Price et al., 2005) were
integrated (although we note that the software allows other gene
association networks to easily be added).

The overall effect of these changes (in addition to other minor
modifications and improvements) resulted in an algorithm run-time
reduction of about 4-fold. This, in turn, enabled \cm\ to be run
numerous times on a modest 8-core compute node (\eg, a c1.xlarge
Amazon EC2 node) in under six hours per complete run (versus the
original \cm\ requiring several days to a week). Practically, the
effect of these modifications to the algorithm resulted in a single
\cm\ run generating, on average, fewer duplicate biclusters,
primarily because each gene is allowed to be a member of only two
biclusters. We found that, in general, this increased the overall
diversity of regulation (conditional clusterings and corresponding
cis-GREs) discovered, per \cm\ run.

\subsubsection{Detailed algorithm description}

\subsubsection{Parameter ranges used for EGRIN 2.0}

The additional parameters used for \cm, and for MEME (which is
used by \cm for motif detection), are the same as those itemized
in (Reiss et al., 2006).

The \cm\ software is available as an open-source R package (Ihaka
and Gentleman, 1996). Using this package, the algorithm can be easily
applied to nearly any sequenced microbial species (given user-supplied
expression data). The package automatically downloads and integrates
genome and annotation data from various external sources, including
RSAtools (van Helden, 2003); MicrobesOnline (Dehal et al., 2010); EMBL
STRING (Szklarczyk et al., 2011), NCBI (Edgar et al., 2002), and KEGG
(Ogata et al., 1999). Additionally, the package can generate
interactive web-based and Cytoscape output (Shannon et al., 2003),
allowing users to explore the resulting modules and motifs in the
context of external data, software, and databases via the Gaggle
(Shannon et al., 2006). Examples of automatically generated output are
available at the \cm\ web site. Supplementary R packages with
example expression data for organisms including \halo\ and
\eco\ are also available from the \cm\ website.

