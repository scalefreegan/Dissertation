\subsection{{\it E. coli} network performance: validation with RegulonDB gold standard}

\subsubsection{Validation of transcription factor binding sites}

We compared the genome-wide locations of predicted GREs (Model
Construction: 5 and 6 above) in the {\it E. coli} ensemble to experimentally
mapped TF binding sites from RegulonDB (BindingSiteSet table, filtered
for experimental evidence and TFs with $\geq 3$ unique binding sites; a
total of 88 TFs). We considered a GRE to be a significant match to a
TF if a significant fraction (FDR $\leq 0.05$) of the predicted non-coding
locations of its PSSMs constituents overlapped with the known binding
locations for that TF (hypergeometric $p$-value $\leq 0.01$; See GRE
definition in Model Construction: 5). In cases where multiple TFs
significantly matched a GRE, only the most significant was
reported. We also observed several instances where more than one GRE
significantly matched the same TF. We were unable to determine whether
this was the result of incomplete GRE clustering, ambiguities related
to GRE scanning, limitations of the experimental data itself, or, by
contrast, a reflection of subtle context-dependent variations in the
binding preferences of these TFs. Since we did not observe clustering
of GREs that map to the same TF upon re-clustering, we hypothesize
that the observations may have biological origins, \ie reflect
condition-dependent variations in TF binding preferences that are the
result, for example, of co-activator/repressor interaction or small
molecule binding. It is interesting to note that TFs with the largest
fraction of GRE matches include transcriptional dual regulators like
FlhDC and UlaR (\ie, TFs with the ability to act as both activators
and repressors). This is consistent with the observation that these
TFs have context-dependent binding preferences. The complete set of
validations, for both TFs and σ-factors, is listed in Table E4.

\paragraph{Comparison with other module detection algorithms}

In addition to the comparisons described above, we also compared the
number of RegulonDB TFs detected in the EGRIN 2.0 model to individual
\cm\ runs as well as several other algorithms that were computed on
subsets of the experimental data (similar to the EGRIN 2.0 ensemble;
Figure E2C). We evaluated: (a) k-means clustering, (b) WGCNA, and (c)
DISTILLER (Lemmens et al., 2009). For (a) and (b), we computed modules
100 times on random subsets of the {\it E. coli} expression data set (using
200-250 randomly chosen experiments per run; selection criteria were
identical to {\it E. coli} EGRIN 2.0). We then predicted de novo
cis-regulatory GREs in the promoter regions of genes in each module
using MEME (MEME parameters were identical to EGRIN 2.0). For (c), we
performed the comparison using the original modules generated by
(Lemmens et al., 2009). Rather than alter module composition by
re-detection, we instead varied MEME parameters applied to the modules
100 times (within the same ranges as those used for EGRIN 2.0). TF-GRE
matches were assigned by comparing GREs to RegulonDB TF binding sites,
as previously described (Model Evaluation and Validation: 1).

We found that individual \cm\ runs discovered a greater number of
RegulonDB binding sites on average than the other methods (41 compared
to 30, 25, and 29 for k-means, WGCNA, and DISTILLER, respectively),
which is consistent with previous findings (Reiss et al.,
2006). Integration of individual \cm\ biclusters into the EGRIN 2.0
ensemble outperformed all individual \cm\ runs. This result is
typical of ensemble-based inference approaches, which supports value
of ensemble integration as part of the EGRIN 2.0 model.

\subsubsection{Comparison with ``direct inference'' networks from CLR and DREAM5}

We subdivided the inferred {\it E. coli} EGRIN 2.0 GRN into two networks:
(1) a GRN derived from Inferelator-predicted transcriptional
influences and (2) a GRN derived from TF-matched GREs detected in gene
promoters (Model Construction: 5 and 6 above). For (1), TF-gene
associations were inferred through TF-bicluster influence (\ie, each
gene in a bicluster was assigned to the TFs inferred to regulate the
bicluster). TF-gene associations were ranked based upon the number of
times that they were observed across the entire EGRIN 2.0
ensemble. The top 100,000 rankings were retained in the final network.

The CLR GRN was computed using default parameters on the same
expression data set as EGRIN 2.0 (number of bins = 10 and spline
degree = 3). Interactions were sorted based on the CLR score. The top
100,000 interactions were retained in the final network. CLR analysis
was performed using MATLAB.

The DREAM5 network was retrieved from (Marbach et al., 2012).

Precision-recall statistics were computed for each of the predicted
{\it E. coli} GRNs using RegulonDB (version 7.2). We used regulatory
interactions annotated as having strong experimental evidence
(Gama-Castro et al., 2011). The resulting gold-standard network
contained 2,427 TF-gene interactions between 155 TFs and 1163
genes. The precision-recall and AUPR statistics were calculated as in
(Marbach et al., 2012).

\subsection{Validation of conditional operons in tiling array transcriptome measurements}

\subsection{Global evaluation of fitness correlations}

We defined gene modules using regulons (annotated in RegulonDB or
RegPrecise) by grouping together genes that were annotated as
controlled by a common TF. Using the same community detection
procedures that we used to define corems from the EGRIN 2.0 ensemble,
we computed gene co-expression modules from the weighted WGCNA
adjacency matrix. We compared the distributions of Pearson
correlations between relative changes in fitness across pairs of genes
within each module, using the one-tailed Kolmogorov-Smirnov test
(KS-test). The precision/recall characteristics for each model are
contained in Table E5.
