\subsubsection{Introduction and summary}

\subsubsection{Input list of regulatory factors}

\subsubsection{Updates since original publication}

While Inferelator (Bonneau et al., 2006) and its more recent
derivatives (Madar et al., 2009) have been successful at accurately
inferring causal regulatory influences using gene expression data, and
predicting global regulatory dynamics in new experiments, the
algorithm as originally published (Bonneau et al., 2006) required
refactoring, and additionally had some notable drawbacks, which were
remedied in a new implementation. Modifications included:

1. Removal of ``pre-grouping'' highly correlated regulators into ``TF
groups'' was removed by replacing the $L_1$ (LASSO) constraint with the
elastic-net linear regression constraint. It has been shown that the
elastic-net constraint results in highly correlated predictors being
grouped as part of the optimization – in this case, using only
conditions relevant for each bicluster – negating the necessity of
pre-grouping the predictors (Zou and Hastie, 2005). We note that in
all Inferelator runs the elastic net parameter value was 0.8 (\ie,
close to the original LASSO $L_1$ constraint (=1), but with a “small
amount” of $L_2$).

2. Elimination of ``pre-filtering'' of regulators for each bicluster based
upon high correlation. The procedure now allows the elastic-net to
choose among all potential regulators (excluding TF members of the
bicluster, which are automatically considered possible regulators,
and are removed from the list of candidate predictors prior to
applying the elastic net).

3. Capability to up-weight measurements. This was utilized in the
EGRIN 2.0 model to up-weight measurements with lower variance (\ie,
more tightly co-expressed) among the genes in a bicluster (\ie,
standard weighted linear least-squares, $w_i=1/\sigma_i^2$).

Additional features, such as the ability to up-weight potential
regulators to increase their likelihood of being selected and
bootstrapping to more robustly select regulator weights are included
in the implementation, but were not specifically utilized as part of
the currently described EGRIN 2.0 models. The current implementation of
Inferelator, which utilizes the elastic net by default is available as
an open-source R package.

\subsubsection{Detailed algorithm description}
