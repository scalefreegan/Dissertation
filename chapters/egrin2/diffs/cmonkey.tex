\subsection{\DIFdelbegin \DIFdel{cMonkey}\DIFdelend \DIFaddbegin \cm\DIFaddend : integrated biclustering algorithm, updated for ensemble analysis}

\subsubsection{Introduction and summary}

The \cm\ integrated biclustering algorithm was described and \DIFdelbegin \DIFdel{benchmarked in (Reiss et al. , 2006). }\DIFdelend \DIFaddbegin \DIFadd{fully
benchmarked in \mbox{%DIFAUXCMD
\cite{Reiss2006n}
}%DIFAUXCMD
. }\DIFaddend In short, the algorithm computes
putatively co-regulated modules of genes over subsets of experimental
conditions from gene expression data, constrained by information
provided by genome sequence (\DIFdelbegin \DIFdel{in the form of de novo }\DIFdelend \DIFaddbegin \textit{\DIFadd{de novo}} \DIFaddend identification of
conserved \DIFdelbegin \DIFdel{cis-regulatory motifs in the }\DIFdelend \DIFaddbegin \textit{\DIFadd{cis}}\DIFadd{-regulatory motifs in }\DIFaddend gene promoters), and functional
association networks. Its defining characteristic is that it
\DIFdelbegin \DIFdel{integrates }\DIFdelend \DIFaddbegin \DIFadd{combines }\DIFaddend all three types of data (expression, sequence and networks)
together into an integrated model that \DIFdelbegin \DIFdel{is optimized via }\DIFdelend \DIFaddbegin \DIFadd{uses }\DIFaddend a stochastic
optimization procedure to identify modules that best satisfy all three
constraints, simultaneously.

\DIFaddbegin \DIFadd{We refer the reader interested in details of the }\cm\ \DIFadd{data integration
model and optimization procedure to \mbox{%DIFAUXCMD
\cite{Reiss2006n}
}%DIFAUXCMD
. Here, we
briefly summarize the procedure as it was utilized in the }\egrine\DIFadd{~
model construction. The }\cm\ \DIFadd{integrated biclustering algorithm
identifies groups of genes co-regulated under subsets of experimental
conditions, by integrating various orthogonal pieces of information
that support evidence for their co-regulation, and optimizing
biclusters such that they are supported by one or more of those
additional constraints. The three sources of evidence for
co-regulation leveraged by cMonkey to score gene clusters are (1)
tight co-expression in subsets of available gene expression
measurements (similarity of expression profiles); (2) quality of
}\textit{\DIFadd{de novo}} \DIFadd{detected }\textit{\DIFadd{cis}}\DIFadd{-regulatory motifs in gene
promoters (putative co-binding of common regulators); and (3)
significant connectivity in functional association or physical
interaction networks (co-functionality). The algorithm served as the
cornerstone for the construction of the first global, predictive
Environmental Gene Regulatory Influence Network (EGRIN) model for
}\halo \DIFadd{\mbox{%DIFAUXCMD
\cite{Bonneau2007}
}%DIFAUXCMD
, and has now been applied to many additional
organisms (}\eg\DIFadd{, \mbox{%DIFAUXCMD
\cite{Yoon2013}
}%DIFAUXCMD
and unpublished).
}

\DIFadd{To run cMonkey as ensemble-based inference approach, we had to make
significant updates to the }\cm\ \DIFadd{algorithm. These updates primarily
addressed computational inefficiencies that led to long runtime. The
primary algorithm modification in the new implementation is global
optimization (rather than the local, individual cluster optimization
utilized by the original procedure). Additional algorithm updates
include changes to the individual scoring scheme for subnetwork
clustering, as well as integration of the scores. All of these changes
improved the procedure's runtime without significantly affecting the
algorithm's performance.
}

\DIFaddend \subsubsection{Updates since original publication}

For incorporation into the \DIFdelbegin \DIFdel{EGRIN 2.0 }\DIFdelend \DIFaddbegin \egrine\DIFadd{~}\DIFaddend ensemble analysis, the \cm\
procedure and software was overhauled to improve \DIFdelbegin \DIFdel{run-time }\DIFdelend \DIFaddbegin \DIFadd{runtime }\DIFaddend performance
and decrease memory usage. These modifications did not quantifiably
affect overall bicluster quality. Changes to the algorithm (and
parameters used for \DIFdelbegin \DIFdel{EGRIN 2.0 }\DIFdelend \DIFaddbegin \egrine\DIFadd{~}\DIFaddend construction) relative to the earlier
version described in \DIFdelbegin \DIFdel{(Reiss et al., 2006) }\DIFdelend \DIFaddbegin \DIFadd{\mbox{%DIFAUXCMD
\cite{Reiss2006n}
}%DIFAUXCMD
}\DIFaddend are as follows:

1. \DIFdelbegin \DIFdel{The use of iteratively }\DIFdelend \DIFaddbegin \DIFadd{Iteratively }\DIFaddend re-weighted constrained logistic regression
to determine gene/condition probabilities for bicluster membership was
replaced with a non-parametric cumulative distribution function on
gene/condition scores. Since the non-parametric function does not need
to be re-weighted, it is significantly faster to compute.

2. Rather than constraining the number of bicluster assignments per
gene/condition using a probability distribution, \cm\DIFaddbegin \ \DIFaddend now assigns a
fixed number of biclusters to each gene/condition, per run (\DIFdelbegin \DIFdel{this is }\DIFdelend a
user-defined parameter\DIFdelbegin \DIFdel{, and for this study }\DIFdelend \DIFaddbegin \DIFadd{). For this study the parameter }\DIFaddend was set to 2 for genes, and
to $k/2$ for conditions, where $k$ is the total number of biclusters in
the run, also a user-defined parameter\DIFdelbegin \DIFdel{)}\DIFdelend . This modification effectively
alters the bicluster optimization from a local
\DIFaddbegin \DIFadd{problem }\DIFaddend (single bicluster) \DIFdelbegin \DIFdel{problem }\DIFdelend with limited cross-bicluster constraints \DIFdelbegin \DIFdel{, }\DIFdelend to a global problem,
similar in principle to the widely used k-means clustering algorithm.

3. Since \cm\ uses the updated constraint \DIFdelbegin \DIFdel{of (see }\DIFdelend \DIFaddbegin \DIFadd{(of }\DIFaddend 2\DIFdelbegin \DIFdel{, }\DIFdelend \DIFaddbegin \DIFadd{; see }\DIFaddend above) to choose
the two ``best'' biclusters for each gene (and \DIFaddbegin \DIFadd{likewise }\DIFaddend the best $k/2$
biclusters for each condition), there is no sampling as in the prior
version. Instead, stochasticity is added \DIFdelbegin \DIFdel{, }\DIFdelend to prevent the optimization
from falling into \DIFdelbegin \DIFdel{a local minimum, by allowing }\DIFdelend \DIFaddbegin \DIFadd{local minima. The algorithm allows }\DIFaddend at most one
change in bicluster assignment per gene/condition, per iteration\DIFdelbegin \DIFdel{, and }\DIFdelend \DIFaddbegin \DIFadd{. This
is accomplished }\DIFaddend by adding a small amount of \DIFdelbegin \DIFdel{artificial }\DIFdelend noise to each
gene/condition's bicluster membership weight. \DIFdelbegin \DIFdel{This }\DIFdelend \DIFaddbegin \DIFadd{The }\DIFaddend noise occasionally
allows moves that decrease a bicluster's total score\DIFdelbegin \DIFdel{, and the noise
decreases to near
zero toward the end of the optimization}\DIFdelend \DIFaddbegin \DIFadd{. This noise
decreases towards zero as the number of iterations increases}\DIFaddend .

4. \DIFdelbegin \DIFdel{The motif search, }\DIFdelend \DIFaddbegin \DIFadd{Motif detection is }\DIFaddend the most computationally expensive part of the
procedure\DIFdelbegin \DIFdel{, is limited to run }\DIFdelend \DIFaddbegin \DIFadd{. To circumvent significant computation time, we limit motif 
detection to }\DIFaddend every 100 iterations (for a typical,
2,000 iteration \cm\ run). During the 99 iterations between motif
searches, the biclusters are optimized to contain instances of those
detected motif(s). We found that this does not impair the ability of
\cm\ to detect \DIFdelbegin \DIFdel{significant }\DIFdelend motifs.

\DIFdelbegin \DIFdel{5. Finally, as part of the EGRIN 2.0 model construction, only the EMBL
STRING (v9) (Szklarczyk et al., 2011) set of predicted gene functional
associations, and predicted operons (Price et al., 2005) were
integrated (although we note that the software allows other gene
association networks to easily be added).
}%DIFDELCMD < 

%DIFDELCMD < %%%
\DIFdelend The overall effect of these changes (in addition to other minor
modifications and improvements) resulted in an algorithm run-time
reduction of about 4-fold. This, in turn, enabled \cm\ to be run
numerous times on a modest 8-core compute node (\eg, a
\DIFdelbegin \DIFdel{c1.xlarge
}\DIFdelend \DIFaddbegin \tmsamp{c1.xlarge} \DIFaddend Amazon EC2 node) in under six hours per 
\DIFdelbegin \DIFdel{complete run (versus the
original }%DIFDELCMD < \cm\ %%%
\DIFdel{requiring }\DIFdelend \DIFaddbegin \DIFadd{run (compared to }\DIFaddend several days to a week \DIFaddbegin \DIFadd{for the original }\cm\ \DIFaddend ). Practically, the effect of these modifications to the algorithm
resulted in a single \cm\ run generating \DIFdelbegin \DIFdel{, on average, }\DIFdelend fewer duplicate
biclusters, primarily because each gene is allowed to be a member of
only two biclusters. We found that \DIFdelbegin \DIFdel{, in general, }\DIFdelend this increased the
overall diversity of regulation \DIFdelbegin \DIFdel{(conditional clusterings and
corresponding cis-GREs)discovered, per }\DIFdelend \DIFaddbegin \DIFadd{discovered by each }\DIFaddend \cm\ \DIFdelbegin \DIFdel{run}\DIFdelend \DIFaddbegin \DIFadd{run (condition-specific gene clusters and
corresponding }\textit{\DIFadd{cis}}\DIFadd{-GREs)}\DIFaddend .

\DIFdelbegin \subsubsection{\DIFdel{Detailed algorithm description}}
%DIFAUXCMD
\addtocounter{subsubsection}{-1}%DIFAUXCMD
\DIFdelend \DIFaddbegin \subsubsection{\DIFadd{Detailed }\cm\ \DIFadd{algorithm description}}
\DIFaddend 

\DIFaddbegin \DIFadd{The }\cm\ \DIFadd{algorithm initiates by seeding $k$ biclusters, typically using
the simple, widely-used and effective $k$-means clustering on the
input expression data set. }\cm\ \DIFadd{itself performs a global optimization,
in many ways similar to the $k$-means clustering algorithm, which we
used as a model. After beginning with an initial assignment of each
gene into $k$ clusters and a chosen distance metric, the basic
$k$-means algorithm iterates between two steps until convergence: (1)
(re-)assign each gene to the cluster with the closest centroid and (2)
update the centroids of each modified cluster. The updated }\cm\ 
\DIFadd{algorithm performs an analogous set of moves with four primary
distinctions: (1) the distance of each gene to the ``centroid'' of
each cluster is computed using a measure that combines
condition-specific expression profile similarity, $cis$-regulatory
motif similarity, and connectedness in one or more gene association
networks; (2) each gene can be (re-)assigned to more than one cluster
(default 2); (3) at each step, conditions (in addition to genes) are
moved among biclusters to improve their cohesiveness; and (4) at each
step, genes and conditions are not always assigned to the most
appropriate clusters. We now elaborate upon these four details.
}

\cm\ \DIFadd{begins each iteration with a set of bicluster memberships
$\{m_i\}$ for each element (gene or condition) $i$, where by default
$\|m_i\| = 2$ for genes and $\|m_i\| = N_c/2$ for conditions ($N_c$ is
the number of conditions, or measurements, in the expression data set;
note that for standard $k$-means clustering, $\|m_i\| = 1$ for genes
and $\|m_i\| = N_c$ for conditions). }\cm\ \DIFadd{then computes score matrices
(log-likelihoods, in practice) $\mathbf{R}_{ij}$, $\mathbf{S}_{ij}$,
and $\mathbf{T}_{ij}$, for membership of each element $i$ in each
bicluster $j$, based upon, respectively, co-expression with the
current gene members ($\mathbf{R}$), similarity of motifs in gene
promoters ($\mathbf{S}$), and connectivity of genes in networks
($\mathbf{T}$). For the network scores ($\mathbf{T}$), the originally
published procedure \mbox{%DIFAUXCMD
\cite{Reiss2006n}
}%DIFAUXCMD
computed a $p$-value for
enrichment of network edges among genes in each bicluster using the
cumulative hypergeometric distribution. This computation was
inefficient, and moreover could not account for weighted edges in the
input networks, so we replaced it with a more standard weighted
network clustering coefficient \mbox{%DIFAUXCMD
\cite{Watts1998}
}%DIFAUXCMD
, evaluated only over
the genes within each bicluster.
}

\DIFadd{Following computation of the individual component scores,
}\cm\ \DIFadd{computes a score matrix $\mathbf{M}_{ij}$ containing the
integrated score (a weighted sum of log-likelihoods) supporting the
inclusion of gene $i$ in bicluster $j$. At this stage the updated
version of }\cm\ \DIFadd{then computes a cumulative density distribution from
each bicluster's $\mathbf{M}_{\cdot j}$ to obtain a posterior
probability distribution $p_{ij}$, that each element $i$ should be in
each cluster $j$, which is used to classify cluster members based upon
these scores. The width of the density distribution kernel is set
dynamically to be larger for smaller (fewer gene) biclusters, so as to
increase the tendency to add genes to small biclusters, rather than to
remove them. In the updated procedure, we then add a small amount of
normally-distributed random ``noise:'' to the scores
$\mathbf{M}_{ij}$, in order to achieve a similar type of stochasticity
to the original version of the algorithm (which was originally
obtained using sampling, and which helps prevent the algorithm from
falling into local minima; this noise decreases during the run to zero
at the final iteration). The result of this noise is that at the
beginning of a }\cm\ \DIFadd{run, biclusters are rather poorly defined
(co-expression, for example, is weak), but during the course of a full
set of 2,000 iterations, as this noise is decreased, the biclusters
settle into minima.
}

\DIFadd{Finally, at the end of each iteration, }\cm\ \DIFadd{chooses a random subset of
elements (genes or conditions) $i$, and moves $i$ into bicluster $j$
if, for any biclusters $j'$ which it is already a member, $p_{ij} >
p_{ij'}, \forall j'$, and out of the corresponding worse bicluster
$j'$ for which $p_{ij} > p_{ij'}$. Thus, as with the $k$-means
clustering algorithm, the updated }\cm\ \DIFadd{performs a global optimization
of all biclusters by moving elements among biclusters to improve each
element's membership scores.
}

\DIFaddend \subsubsection{Parameter ranges used for \DIFdelbegin \DIFdel{EGRIN 2.0}\DIFdelend \DIFaddbegin \egrine\DIFaddend }

The \DIFaddbegin \DIFadd{default values for all }\DIFaddend additional parameters used for \cm, and for
\DIFdelbegin \DIFdel{MEME }\DIFdelend \DIFaddbegin \MEME\ \DIFaddend (which is used by \cm\DIFaddbegin \ \DIFaddend for motif detection\DIFaddbegin \DIFadd{; \mbox{%DIFAUXCMD
\cite{Bailey1998}
}%DIFAUXCMD
}\DIFaddend ),
are the same as those itemized in \DIFdelbegin \DIFdel{(Reiss et al. , 2006) }\DIFdelend \DIFaddbegin \DIFadd{\mbox{%DIFAUXCMD
\cite{Reiss2006n}
}%DIFAUXCMD
. These defaults
were used exclusively for all }\halo\DIFadd{~ }\cm\DIFadd{~ runs. These default
parameters are itemized in Table~\ref{tab:cmparams:halo}. The only
parameter that varied from run-to-run for the }\halo\DIFadd{~ }\cm\DIFadd{~ runs was the
number of conditions (columns in the expression matrix) included. As
}\cm\DIFadd{~development was occurring in parallel to development of the
}\egrine\DIFadd{~modeling approach (primarily involving bug fixes), we also
list the official }\cm\DIFadd{~version number(s) used}\DIFaddend .

\DIFaddbegin \vspace{.2cm}
\begin{table}[h!]
\begin{tabular}{|l|l|l|r|} 
\hline
\DIFaddFL{Parameter              }& \DIFaddFL{Value              }& \DIFaddFL{Note }\\ \hline
\DIFaddFL{Version               }& \DIFaddFL{4.5.4(174), 4.6(191), 4.6.2(109)        }& \cm\DIFaddFL{~package versions (and number of runs) used }\\
\DIFaddFL{$N_{\mathrm{conds}}$    }& \DIFaddFL{242:300          }& \DIFaddFL{Number of conditions included (range) }\\
\DIFaddFL{$k$                   }& \DIFaddFL{250              }& \DIFaddFL{Number of biclusters }\\
\DIFaddFL{$N_{\mathrm{gene}}$    }& \DIFaddFL{2          }& \DIFaddFL{Number of biclusters per gene }\\
\DIFaddFL{$N_{\mathrm{iter}}$     }& \DIFaddFL{2000               }& \DIFaddFL{Number of iterations }\\
\DIFaddFL{$w_{\mathrm{max}}$     }& \DIFaddFL{24               }& \DIFaddFL{Maximum }\MEME\DIFaddFL{~motif width }\\
\DIFaddFL{$w_{\mathrm{min}}$     }& \DIFaddFL{6               }& \DIFaddFL{Minimum }\MEME\DIFaddFL{~motif width }\\
\DIFaddFL{$l_{\mathrm{search}}$     }& \DIFaddFL{-150 -- +20       }& \MEME\DIFaddFL{~search distance from gene start }\\
\DIFaddFL{$l_{\mathrm{scan}}$     }& \DIFaddFL{-250 -- +30       }& \MEME\DIFaddFL{~scan distance from gene start }\\
\DIFaddFL{$n_{\mathrm{motif}}$     }& \DIFaddFL{2       }& \DIFaddFL{Number of }\MEME\DIFaddFL{~motifs searched per bicluster }\\
\DIFaddFL{$s_{\mathrm{r}}$     }& \DIFaddFL{1       }& \DIFaddFL{Scaling factor for expression scores }\\
\DIFaddFL{$s_{\mathrm{m}}$     }& \DIFaddFL{1       }& \DIFaddFL{Scaling factor for motif scores }\\
\DIFaddFL{$s_{\mathrm{n}}$     }& \DIFaddFL{0.5       }& \DIFaddFL{Scaling factor for network scores }\\
\DIFaddFL{$w_{\mathrm{op}}$     }& \DIFaddFL{0.5       }& \DIFaddFL{Relative weight for operon network }\\
\DIFaddFL{$w_{\mathrm{string}}$     }& \DIFaddFL{0.5       }& \DIFaddFL{Relative weight for STRING network }\\
\hline
\end{tabular}
\caption{\cm\DIFaddFL{~parameters used for the }\halo\DIFaddFL{~ensemble.}}
\label{tab:cmparams:halo}
\end{table}
\vspace{.2cm}

\noindent 
\DIFadd{In contrast to the }\halo\DIFadd{~ runs, for the }\eco\DIFadd{~ runs, we varied 
several parameters randomly between the ranges itemized in Table~\ref{tab:cmparams:eco}.
}

\vspace{.2cm}
\begin{table}[h!]
\begin{tabular}{|l|l|l|r|} 
\hline
\DIFaddFL{Parameter              }& \DIFaddFL{Value              }& \DIFaddFL{Note }\\ \hline
\DIFaddFL{Version               }& \DIFaddFL{4.9.0(106)        }& \cm\DIFaddFL{~package versions (and number of runs) used }\\
\DIFaddFL{$N_{\mathrm{conds}}$    }& \DIFaddFL{181:279          }& \DIFaddFL{Number of conditions included (range) }\\
\DIFaddFL{$k$                   }& \DIFaddFL{352:547              }& \DIFaddFL{Number of biclusters }\\
\DIFaddFL{$N_{\mathrm{gene}}$    }& \DIFaddFL{2          }& \DIFaddFL{Number of biclusters per gene }\\
\DIFaddFL{$N_{\mathrm{iter}}$     }& \DIFaddFL{2000               }& \DIFaddFL{Number of iterations }\\
\DIFaddFL{$w_{\mathrm{max}}$     }& \DIFaddFL{12:30               }& \DIFaddFL{Maximum }\MEME\DIFaddFL{~motif width }\\
\DIFaddFL{$w_{\mathrm{min}}$     }& \DIFaddFL{6               }& \DIFaddFL{Minimum }\MEME\DIFaddFL{~motif width }\\
\DIFaddFL{$l_{\mathrm{search}}$     }& \DIFaddFL{-(100:200) -- +(0:20)       }& \MEME\DIFaddFL{~search distance from gene start }\\
\DIFaddFL{$l_{\mathrm{scan}}$     }& \DIFaddFL{-(-150:250) -- +(0:50)       }& \MEME\DIFaddFL{~scan distance from gene start }\\
\DIFaddFL{$n_{\mathrm{motif}}$     }& \DIFaddFL{1:3       }& \DIFaddFL{Number of }\MEME\DIFaddFL{~motifs searched per bicluster }\\
\DIFaddFL{$s_{\mathrm{r}}$     }& \DIFaddFL{2:4       }& \DIFaddFL{Scaling factor for expression scores }\\
\DIFaddFL{$s_{\mathrm{m}}$     }& \DIFaddFL{0.5       }& \DIFaddFL{Scaling factor for motif scores }\\
\DIFaddFL{$s_{\mathrm{n}}$     }& \DIFaddFL{0.5       }& \DIFaddFL{Scaling factor for network scores }\\
\DIFaddFL{$w_{\mathrm{op}}$     }& \DIFaddFL{0:1       }& \DIFaddFL{Relative weight for operon network }\\
\DIFaddFL{$w_{\mathrm{string}}$     }& \DIFaddFL{0:1       }& \DIFaddFL{Relative weight for STRING network }\\
\hline
\end{tabular}
\caption{{\cm}\DIFaddFL{~parameters used for the }\eco\DIFaddFL{~ensemble.}}
\label{tab:cmparams:eco}
\end{table}
\vspace{.2cm}

\subsubsection{\cm\ \DIFadd{software availability}}

\DIFaddend The \cm\ software is available as an open-source \DIFdelbegin \DIFdel{R package
(Ihaka
and Gentleman, 1996).  Using this package , }\DIFdelend \DIFaddbegin \tmsamp{R} \DIFadd{package
\mbox{%DIFAUXCMD
\cite{Ihaka1996}
}%DIFAUXCMD
.  With this package }\DIFaddend the algorithm can be easily
applied to nearly any sequenced microbial species (given user-supplied
expression data). The package automatically downloads and integrates
genome and annotation data from various external sources, including
\DIFdelbegin \DIFdel{RSAtools (van Helden, 2003); MicrobesOnline (Dehal et al., 2010); EMBL
STRING (Szklarczyk et al., 2011), NCBI (Edgar et al., 2002), and KEGG
(Ogata et al. , 1999). }\DIFdelend \DIFaddbegin \tmsamp{RSA-tools} \DIFadd{\mbox{%DIFAUXCMD
\cite{vanHelden2000}
}%DIFAUXCMD
; }\tmsamp{Microbes Online}
\DIFadd{\mbox{%DIFAUXCMD
\cite{Alm2005}
}%DIFAUXCMD
; and }\tmsamp{EMBL STRING}
\DIFadd{\mbox{%DIFAUXCMD
\cite{Szklarczyk2011}
}%DIFAUXCMD
. }\DIFaddend Additionally, the package can generate
interactive web-based and \DIFdelbegin \DIFdel{Cytoscape output
(Shannon et al., 2003),
}\DIFdelend \DIFaddbegin \tmsamp{Cytoscape} \DIFadd{output
\mbox{%DIFAUXCMD
\cite{Shannon2003}
}%DIFAUXCMD
, }\DIFaddend allowing users to explore the resulting modules
and motifs in the context of external data, software, and databases
via the \DIFdelbegin \DIFdel{Gaggle
(Shannon et al. , 2006). }\DIFdelend \DIFaddbegin \tmsamp{Gaggle} \DIFadd{\mbox{%DIFAUXCMD
\cite{Shannon2006}
}%DIFAUXCMD
. }\DIFaddend Examples of automatically
generated output are available at the \cm\ web site. Supplementary \DIFdelbegin \DIFdel{R }\DIFdelend \DIFaddbegin \DIFadd{$R$
}\DIFaddend packages with example expression data for organisms including
\halo\ and \eco\ are also available from the \cm\ website.

