\subsubsection{Introduction and summary}

\DIFdelbegin \subsubsection{\DIFdel{Input list of regulatory factors}}
%DIFAUXCMD
\addtocounter{subsubsection}{-1}%DIFAUXCMD
\DIFdelend \DIFaddbegin \DIFadd{The }\nwinf\ \DIFadd{algorithm is a method for deriving
genome-wide transcriptional regulatory interactions from mRNA
expression levels \mbox{%DIFAUXCMD
\cite{Bonneau2006}
}%DIFAUXCMD
. }\nwinf\ \DIFadd{is a direct inference
procedure \mbox{%DIFAUXCMD
\cite{Michoel2009}
}%DIFAUXCMD
. It models transcriptional regulation
as a kinetic process, incorporating time information, when available,
and a user-defined time constant. }\nwinf\ \DIFadd{uses standard regression and
variable selection to identify transcriptional influences on genes or
biclusters based on their mean expression levels. These influences include expression
levels of TFs, environmental factors, and interactions between the two. The procedure
simultaneously models equilibrium and time course expression levels. Thus both kinetic and equilibrium expression levels may be
predicted by the resulting models. Through  explicit inclusion of
time and gene knockout information, the method is capable of learning
causal relationships. The inferred network is a predictive model
comprised of linear combinations of expression profiles of various
transcriptional regulators, that can predict global expression under
novel perturbations with predictive power similar to that seen over
training data \mbox{%DIFAUXCMD
\cite{Bonneau2006}
}%DIFAUXCMD
.
}\DIFaddend 

\subsubsection{Updates since original publication}

\DIFdelbegin \DIFdel{While Inferelator (Bonneau et al. , 2006) and its more recent
derivatives (Madar et al. , 2009) have been successful at accurately
inferring causal regulatory influences using gene expression data, and
predicting global regulatory dynamics in new experiments, the
algorithm as originally published (Bonneau et al. , 2006) required
refactoring, and additionally had some notable drawbacks, which were
remedied in a new implementation.
Modifications included:
}\DIFdelend \DIFaddbegin \DIFadd{Several modifications have been made to improve the originally published }\nwinf\ \DIFadd{algorithm \mbox{%DIFAUXCMD
\cite{Bonneau2006}
}%DIFAUXCMD
. 
}\DIFaddend 

1. \DIFdelbegin \DIFdel{Removal of ``}\DIFdelend \DIFaddbegin \DIFadd{Elimination of }\DIFaddend pre-grouping \DIFdelbegin \DIFdel{'' }\DIFdelend highly correlated regulators
into ``TF groups''\DIFdelbegin \DIFdel{was removed }\DIFdelend \DIFaddbegin \DIFadd{. This step became obsolete
}\DIFaddend by replacing the $L_1$ (LASSO) constraint with the elastic-net
linear regression constraint. It has been shown that the elastic-net
constraint results in highly correlated predictors being grouped as
part of the optimization\DIFdelbegin \DIFdel{– in this case, using only
conditions relevant for each bicluster – negating }\DIFdelend \DIFaddbegin \DIFadd{. This relieves }\DIFaddend the necessity of pre-grouping
\DIFdelbegin \DIFdel{the predictors (Zou and Hastie, 2005)}\DIFdelend \DIFaddbegin \DIFadd{predictors \mbox{%DIFAUXCMD
\cite{Zou05regularizationand}
}%DIFAUXCMD
}\DIFaddend . We note that in
all \DIFdelbegin \DIFdel{Inferelator runs the elastic net parameter value was 0.8 }\DIFdelend \DIFaddbegin \nwinf\ \DIFadd{runs the elastic-net parameter value $\alpha$ was set to
$\alpha=0.8$ }\DIFaddend (\ie, close to the original LASSO $L_1$ constraint
(\DIFdelbegin \DIFdel{=1}\DIFdelend \DIFaddbegin \DIFadd{corresponding to $\alpha=1$}\DIFaddend ), but with a \DIFdelbegin \DIFdel{“small amount�}\DIFdelend \DIFaddbegin \DIFadd{``small amount'' }\DIFaddend of
$L_2$). \DIFaddbegin \DIFadd{We used the elastic-net implementation provided by
the }\tmsamp{R} \texttt{\DIFadd{glmnet}}
\DIFadd{package \mbox{%DIFAUXCMD
\cite{Friedman:Hastie:Tibshirani:2009:JSSOBK:v33i01}
}%DIFAUXCMD
}\DIFaddend 

2. Elimination of \DIFaddbegin \DIFadd{the }\DIFaddend ``pre-filtering'' of regulators for each
bicluster based upon high correlation. The procedure now allows the
elastic-net to choose among all potential regulators (excluding TF
members of the bicluster, which are automatically considered possible
regulators, and are removed from the list of candidate predictors
prior to applying the \DIFdelbegin \DIFdel{elastic net}\DIFdelend \DIFaddbegin \DIFadd{elastic-net}\DIFaddend ).

3. Capability to up-weight measurements. This was utilized in the
\DIFdelbegin \DIFdel{EGRIN 2.0 }\DIFdelend \DIFaddbegin \egrine\DIFadd{~}\DIFaddend model to up-weight measurements with lower variance (\ie,
more tightly co-expressed) among the genes in a bicluster \DIFdelbegin \DIFdel{(}%DIFDELCMD < \ie%%%
\DIFdel{,
}\DIFdelend \DIFaddbegin \DIFadd{by
}\DIFaddend standard weighted linear least-squares, $w_i=1/\sigma_i^2$\DIFdelbegin \DIFdel{)}\DIFdelend . 

\DIFdelbegin \DIFdel{Additional features, such asthe ability to up-weight potential
regulators to increase their likelihood of being selected and bootstrapping to more robustly select regulator weights are included
in the implementation, but were not specifically utilized as part of
the currently described EGRIN 2.0 models. The current implementationof
Inferelator, which utilizes the elastic net by
default is available as an }\DIFdelend \DIFaddbegin \DIFadd{The current implementation of
}\nwinf\ \DIFadd{is available as
an }\DIFaddend open-source \DIFdelbegin \DIFdel{R package. }\DIFdelend \DIFaddbegin \tmsamp{R} \DIFadd{package.
}\DIFaddend 

\subsubsection{Detailed algorithm description}
\DIFaddbegin 

\DIFadd{Given an input list of $p$ putative transcriptional influences
$\mathbf{X}={x_1,x_2,\cdots,x_p}$ and the mean expression levels $y_{i}$
of a bicluster $k$ (over the conditions $i$ included in the bicluster), we
model the relationship between $y_{i}$ and the influences
$\mathbf{X}$ by the kinetic equation:
}

\begin{equation}\DIFadd{
\label{eq:nwinf:kinetic}
\tau\frac{dy_{i}}{dt}=-y_{i}+\sum_{j=1}^{p}\beta_j x_{ij} .
}\end{equation}

\noindent \DIFadd{In the steady state scenario, $dy/dt=0$ and Eq.~\ref{eq:nwinf:kinetic} simplifies to
}

\begin{equation*}\DIFadd{
y_{i}=\sum_{j=1}^{p}\beta_j x_{ij} ,
}\end{equation*}

\noindent \DIFadd{and for time series measurements, we approximate Eq.\ref{eq:nwinf:kinetic} as:
}

\begin{equation*}\DIFadd{
\tau\frac{y_{i+1}-y_{i}}{t_{i+1}-t_i} + y_{i} = \sum_{j=1}^{p}\beta_j x_{ij} .
}\end{equation*}

\noindent \DIFadd{Clearly not all $p$ putative influences $\mathbf{X}$ influence a given
bicluster, so we use the elastic-net \mbox{%DIFAUXCMD
\cite{Zou05regularizationand}
}%DIFAUXCMD
for variable selection. This involves performing the minimization:
}

\begin{equation}\DIFadd{
\label{eq:elnet}
 \vec{\beta} = \arg \min \left\{ \sum^N_{i = 1} \left( y_i - \sum^p_{j = 1} \beta_j x_{ij} \right)^2 + \sum_{j = 1}^p
   \tmmathbf{w}_i ( \lambda_1 | \beta_j | + \lambda_2 \beta^2_j) \right\}
}\end{equation}

\noindent \DIFadd{subject to a constraint which is a tuneable combination of the $L_1$
(LASSO) and $L_2$ (Ridge) regression constraints:
}

\vspace{-0.4in}

\begin{eqnarray*}\DIFadd{
\label{eq:elnet:constraint}
\sum_{j = 1}^p | \beta_j | \leq \lambda_1 | \beta_{\tmop{ols}} | }& & \DIFadd{\mathrm{(}L_1\ \mathrm{constraint)}, }\\
\DIFadd{\sum_{j = 1}^p \beta^2_j \leq \lambda_2 \beta^2_{\tmop{ols}}     }& & \DIFadd{\mathrm{(}L_2\ \mathrm{constraint)}.
}\end{eqnarray*}

\noindent 
\DIFadd{The $w_i$ in Eq.~\ref{eq:elnet} allow different variables ($\beta$'s,
in this case) to be selectively constrained. For this work, we set all
$w_i=1$, }\ie\DIFadd{, no differential constraints. Redefining the constraint, such that:
}

\begin{equation}\DIFadd{
\label{eq:elnet2}
 \vec{\beta} = \arg \min \left\{ \sum^N_{i = 1} \left( y_i - \sum^p_{j = 1} \beta_j x_{ij} \right)^2 + \sum_{j = 1}^p
   \tmmathbf{w}_i \lambda \left( \alpha | \beta_j | + (1-\alpha) \beta^2_j /2\right) \right\}
}\end{equation}

\noindent \DIFadd{defines $0\leq \alpha \leq 1$ as a tuning parameter between the ridge
($L_2$; $\alpha=0$) and LASSO ($L_1$; $\alpha=1)$ solutions, and
$\lambda$ is the single complexity parameter, which is chosen to
minimize the cross-validation error (we use 10-fold cross-validation),
exactly as in \mbox{%DIFAUXCMD
\cite{Bonneau2006}
}%DIFAUXCMD
. Substituting
Eq.~\ref{eq:nwinf:kinetic} into Eq.~\ref{eq:elnet2}, we obtain the complete
equation describing the minimization performed by }\nwinf\DIFadd{:
}

\begin{equation}\DIFadd{
\label{eq:elnet3}
 \vec{\beta} = \arg \min \left\{ \sum^N_{i = 1} 
 \left( \tau\frac{y_{i+1}-y_{i}}{t_{i+1}-t_i} + y_{i}- \sum_{j=1}^{p} \beta_j x_{i,j} \right)^2 + \sum_{j = 1}^p
   \tmmathbf{w}_i \lambda \left( \alpha | \beta_j | + (1-\alpha) \beta^2_j /2\right) \right\}.
}\end{equation}

\noindent \DIFadd{For the current implementation, we set
$\tau=10$ minutes for all TFs, and $\alpha=0.8$ for all biclusters. In
the future, we could choose $\tau$ and/or $\alpha$ by
cross-validation as well. When $\alpha=0$, there is no constraint, and
we get the ordinary least-squares (OLS) solution with all $\beta$s
non-zero. With $\alpha=1$, we select the null model. The optimal
solution is somewhere in-between, and this is usually the selected
solution for each bicluster, usually $\sim 6$ TFs, on average;
although the null model (no solution) is selected for a number of
biclusters.
 }\DIFaddend