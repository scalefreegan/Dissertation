
\subsubsection{\halo\DIFaddbegin \DIFadd{~ compendium}\DIFaddend }\label{halodata}

A compendium of 1495 transcriptome profiles were collated from a wide
array of experiments conducted by our lab \DIFaddbegin \DIFadd{over the past decade }\DIFaddend that
cover dynamic transcriptional responses to varied growth \DIFaddbegin \DIFadd{(1159 arrays), nutritional (161 arrays)}\DIFaddend ,
\DIFdelbegin \DIFdel{nutritional, }\DIFdelend and stress conditions \DIFaddbegin \DIFadd{(1102 arrays), including variation in temperature (256 arrays), 
oxygen (285 arrays), light (786 arrays), salinity (20 arrays), 
metal ions (274 arrays)}\DIFaddend , \DIFdelbegin \DIFdel{temperatures, 
salinities, metal ions, }\DIFdelend and genetic perturbations (\DIFdelbegin \DIFdel{full set of annotations available online)}\DIFdelend \DIFaddbegin \DIFadd{643 arrays). 
We categorized the experiments using extensive metadata collected at
the time of the experiment. We used this metadata to construct a GO-like 
ontology of the relationships between all experiments (discussed in detail below).
The annotation counts above are derived from this resource (note that a single array can 
receive more than one annotation).
A full list of the metadata, annotations, and ontology is available on the web service}\DIFaddend .
1159 of \DIFdelbegin \DIFdel{these }\DIFdelend \DIFaddbegin \DIFadd{the arrays }\DIFaddend are published (\DIFdelbegin \DIFdel{Baliga et al. , 2004; Baliga et al., 2002; Bonneau
et al., 2007; Facciotti et al., 2010; Facciotti et al., 2007; Kaur et
al., 2006; Kaur et al., 2010; Schmid et al., 2011; Schmid et al.,
2007; Schmid et al., 2009; Whitehead et al., 2006; Whitehead et al.,
2009). }\DIFdelend \DIFaddbegin \DIFadd{\mbox{%DIFAUXCMD
\cite{Baliga2004a,Baliga2002,Bonneau2007,Facciotti2010,Facciotti2007a,Kaur2006,Kaur2010,Schmid2011,Schmid2007,Schmid2009,Whitehead2006,Whitehead2009}
}%DIFAUXCMD
. }\DIFaddend 336 \DIFaddbegin \DIFadd{of the arrays }\DIFaddend are new for this study. Experimental protocols are
identical to \DIFdelbegin \DIFdel{(Bonneau et al. , 2007). }\DIFdelend \DIFaddbegin \DIFadd{\mbox{%DIFAUXCMD
\cite{Bonneau2007}
}%DIFAUXCMD
. }\DIFaddend These data, including expression
levels (log$_2$ ratios vs. reference samples) and experimental
metadata, are available \DIFaddbegin \DIFadd{online }\DIFaddend as a tab-delimited spreadsheet.

\DIFdelbegin \paragraph{\DIFdel{Data normalization}} %DIFAUXCMD
\addtocounter{paragraph}{-1}%DIFAUXCMD
%DIF < % note this probably doesnt need to be a separate section
\DIFdelend %DIF > \paragraph{Data normalization} %% note this probably doesnt need to be a separate section

\DIFdelbegin \subsubsection{%DIFDELCMD < \eco%%%
}%DIFAUXCMD
\addtocounter{subsubsection}{-1}%DIFAUXCMD
\DIFdelend \DIFaddbegin \DIFadd{Each array in the }{\it \DIFadd{H. salinarum}} \DIFadd{compendium was collected using
the same platform, using the same reference, and processed and
normalized using the same protocol. More specifically, each RNA sample
was hybridized along with a }{\it \DIFadd{H. salinarum NRC-1}} \DIFadd{reference RNA
prepared under standard conditions (mid-logarithmic phase batch
cultures grown at 37$^{\circ}$C in CM, OD = 0.5). Samples were hybridized to a
70-mer oligonucleotide array containing the 2400 nonredundant open
reading frames (ORFs) of the }{\it \DIFadd{H. salinarum NRC-1}} \DIFadd{genome as
described in \mbox{%DIFAUXCMD
\cite{Baliga2004a}
}%DIFAUXCMD
. Each ORF was spotted on each array
in quadruplicate and dye flipping was conducted (to rule out bias in
dye incorporation) for all samples, yielding eight technical
replicates per gene per sample. At least two independent biological
replicates exist for all experimental conditions for a total of 16
replicates per gene per condition. Direct RNA labeling, slide
hybridization, and washing protocols were performed as described by
\mbox{%DIFAUXCMD
\cite{Facciotti2007,Schmid2007}
}%DIFAUXCMD
. Raw intensity signals
from each slide were processed by the SBEAMS-microarray pipeline
\mbox{%DIFAUXCMD
\cite{Marzolf2006a}
}%DIFAUXCMD
(www.SBEAMS.org/microarray), in which the
data were median normalized and subjected to significant analysis of
microarrays (SAM) and variability and error estimates analysis
(VERA). Each data point was assigned a significance statistic,
$\lambda$, using maximum likelihood \mbox{%DIFAUXCMD
\cite{Ideker2000}
}%DIFAUXCMD
.
}

\subsubsection{\eco\DIFadd{~ expression compendia}}\DIFaddend \label{ecodata}

\DIFdelbegin \DIFdel{The }\DIFdelend \DIFaddbegin \paragraph{\DIFadd{Use of the }\tmsamp{DISTILLER} \DIFadd{data compendium for model training}}

\DIFadd{A total of 868 }\DIFaddend \eco\DIFdelbegin \DIFdel{data set was obtained from the DISTILLER website (Lemmens
et al., 2009) . }\DIFdelend \DIFaddbegin \ \DIFadd{transcriptome profiles were compiled by
\mbox{%DIFAUXCMD
\cite{Lemmens2009a}
}%DIFAUXCMD
for use with their }\tmsamp{DISTILLER} \DIFadd{algorithm. 
}\DIFaddend These data were collated from publicly available microarray \DIFdelbegin \DIFdel{data consisting of 3 major microarray databases:
}\DIFdelend \DIFaddbegin \DIFadd{databases:
44 arrays from }\DIFaddend Stanford Microarray Database \DIFdelbegin \DIFdel{(Demeter et al.,
2007), }\DIFdelend \DIFaddbegin \DIFadd{\mbox{%DIFAUXCMD
\cite{Demeter2007d}
}%DIFAUXCMD
,
617 from }\DIFaddend Gene Expression Omnibus \DIFdelbegin \DIFdel{(Barrett et al., 2007) and ArrayExpress (Parkinson et al., 2007), }\DIFdelend \DIFaddbegin \DIFadd{\mbox{%DIFAUXCMD
\cite{Barrett2007}
}%DIFAUXCMD
and 36 from
ArrayExpress \mbox{%DIFAUXCMD
\cite{Parkinson2007}
}%DIFAUXCMD
, as well as 181 arrays from
supplementary data in literature (for four different experiments). }\DIFaddend The
experiments cover a range of conditions, including varying carbon
sources \DIFdelbegin \DIFdel{, pH, oxygen, metals and temperature.
}\DIFdelend \DIFaddbegin \DIFadd{(136 arrays), pH (46 arrays), oxygen (284 arrays), metals (27
arrays) and temperature (23 arrays). Overall, the compendium consists
of measurements from single channel (407 arrays; including 298 Affymetrix, 
and 109 P33) and dual channel (460 arrays; including 337 DNA/cDNA
and 126 oligonucleotide) platforms.
}\DIFaddend 

%DIF > \paragraph{Data normalization}
\DIFaddbegin 

\DIFadd{These microarray measurements were normalized by the
authors \mbox{%DIFAUXCMD
\cite{Lemmens2009a}
}%DIFAUXCMD
, as follows: ``If possible, raw
intensities were preferred as data source over normalized data
provided by the public repository. Dual-channel data were loess fitted
to remove nonlinear, dye-related discrepancies. No background
correction procedures were performed to avoid an increase in
expression logratio variance for lower, less reliable intensity
levels. Whenever raw data were available, single-channel data were
first normalized per experiment with RMA. Logratios were then
created for the single-channel data in order to combine them with the
dual channel measurements. For each single-channel array, expression
logratios were computed by comparing the normalized values against an
artificial reference array.  This artificial reference array was
constructed on a per experiment basis by taking the median expression
of each gene across all arrays in the corresponding experiment. When
deemed necessary (e.g. experiments normalized by MAS5.0 for which the
raw data was not available), a loess fit was performed on these
logratios. To ensure that the artificial reference was not altered by
this intensity dependent non-linear rescaling, the artificial
reference expression levels were chosen for the average log intensity
(instead of the mean expression levels of the respective array and the
artificial reference). To ensure comparability between arrays with a
different reference, gene expression profiles were median centered
across arrays that share the same reference. An additional variance
rescaling of the gene expression profiles was performed to render
genes with differing magnitudes of expression changes more
comparable.''
}

\DIFadd{The authors further note that,
``the array composition of the modules generated by }\tmsamp{DISTILLER}
\DIFadd{is not biased towards arrays from any specific platform, indicating a
correct preprocessing of the microarray compendium.'' \mbox{%DIFAUXCMD
\cite{Lemmens2009a}
}%DIFAUXCMD
It is for this
reason that we chose this normalized }{\it
\DIFadd{E. coli}} \DIFadd{microarray compendium for }\egrine\DIFadd{~analysis.
}

\DIFaddend \paragraph{\DIFdelbegin \DIFdel{Data normalization}\DIFdelend \DIFaddbegin \DIFadd{Use of the }\tmsamp{DREAM5} \DIFadd{data compendium for model validation}\DIFaddend }
\DIFaddbegin \label{section:dream5_data_compendium}

\DIFadd{To ascertain the generalizability of }\egrine\DIFadd{~models across data sets, we inferred a second }\textit{\DIFadd{E. coli}} \egrine\DIFadd{~model on an independent }\textit{\DIFadd{E. coli}} \DIFadd{gene expression compendium. By comparing this model to the original model we inferred using the }\tmsamp{DISTILLER} \DIFadd{data set, we were able (1) to  understand what, if any, systematic biases exist due to normalization procedures, and (2) to cross-validate }\egrine\DIFadd{~predictions across two data set. Detailed discussion of the results from this analysis are provided in Section~\ref{sec:validation}.
}

\DIFadd{We obtained the de-anonymized }{\it \DIFadd{E. coli}} \DIFadd{microarray compendium from the }\tmsamp{DREAM5} \DIFadd{competition website \mbox{%DIFAUXCMD
\cite{Marbach2012}
}%DIFAUXCMD
. According to the authors, these data were ``compiled for }{\it \DIFadd{E. coli}}\DIFadd{, where all chips are the same
Affymetrix platform, the }\textit{\DIFadd{E. coli}} \DIFadd{Antisense Genome Array. Chips were
downloaded from GEO (Platform ID: GPL199). In total, 805 chips with
available raw data Affymetrix files (.CEL files) were compiled.''
Additionally, ``Microarray normalization was done using Robust
Multichip Averaging (RMA) 9 through the software RMAExpress. All 160
chips were uploaded into RMAExpress and normalization was done as one
batch. All arrays were background adjusted, quantile normalized, and
probesets were summarized using median polish. Normalized data was
exported as log-transformed expression values. Mapping of Affymetrix
probeset ids to gene ids was done using the library files made
available from Affymetrix. Control probesets and probesets that did
not map unambiguously to one gene were removed, specifically probeset
ids ending in }\_x\DIFadd{, }\_s\DIFadd{, }\_i \DIFadd{were removed. Lastly, if multiple
probesets mapped to a single gene, then expression values were
averaged within each chip.''
}

\DIFadd{Compared to the }\tmsamp{DISTILLER} \DIFadd{\mbox{%DIFAUXCMD
\cite{Lemmens2009a}
}%DIFAUXCMD
data set, the }\tmsamp{DREAM5} \DIFadd{\mbox{%DIFAUXCMD
\cite{Marbach2012}
}%DIFAUXCMD
compendium contained a different subset of the available }\textit{\DIFadd{E. coli}} \DIFadd{transcriptome measurements from a different combination of platforms. While one might expect a number of arrays to be common between the two compendia, we discovered that the two data sets differed substantially in their statistical properties. The maximum Pearson correlation between arrays across the two data sets, for example, was $\sim 0.63$.  Interestingly, the correlation among expression profiles of
genes within predicted operons \mbox{%DIFAUXCMD
\cite{Price2005a}
}%DIFAUXCMD
was higher in the
}\tmsamp{DREAM5} \DIFadd{compendium (mean $\sim 0.83$) than the }\tmsamp{DISTILLER} \DIFadd{compendium
(mean $\sim 0.32$). This is likely due to a combination of differences in the experiments/platforms included and normalization procedures. 
 }\DIFaddend